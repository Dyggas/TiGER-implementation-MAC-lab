\chapter{\large{Вступне слово}}
\label{chapter:1}

\textbf{Мета роботи} (власне, для чого ми тут зібралися): 

Дослідити особливостей реалізації сучасних алгебраїчних криптосистем на прикладі учасників першого раунду 
національного конкурсу з постквантової криптографії в Кореї (KpqC).

\noindent \textbf{Наші задачі на комп'ютерний практикум та порядок їх виконання:}
\begin{enumerate}[label=\arabic*)]
    \item Роздітися на бригади. Визначили хто за що відповідатиме. Богдан -- займається реалізацією алгоритму TiGER, 
        Олексій -- теоретичною частиною і звітом загалом.
    \item Провести теоретичне дослідження теми, надавши вичерпний та повний опис теоретичної сторони алгоритму з усіма 
        деталями та відомими результатами досліджень; провести аналіз вже існуючих атак на алгоритм TiGER, а також 
        загалом можливих атак; виконати порівняльний аналіз нашого алгоритму зi схожими та дослідити можливість 
        перенесення та застосування відомих атак на нього.
    \item Реалізувати алгоритм програмно та всі(нє, ну ми постараємося) можливі варіанти цього алгоритму;
    \item Перевірити коректність -- підтвердити правильність реалізації за допомогою тестів, використавши тестові 
        дані з офіційної реалізації;
    \item Зробити аналіз продуктивності алгоритму та, знову ж таки, провести порівняння та аналіз швидкодії за різних 
        умов, дослідити вплив модифікацій окремих його складових частин на ефективність.
\end{enumerate}

\chapter{\large{Загальне теоретичне дослідження}}
\label{chapter:2}
\section{Постквантова криптографія}
\label{sec:2-1}

Сучасна криптографія з відкритим ключем, зокрема RSA та криптографія на еліптичних кривих -- Elliptic Curve 
Cryptography (ECC), базується на обчислювальній складності задач факторизації великих чисел та дискретного логарифмування. 
Однак у 1994 році Пітер Шор~\cite{shor1994} продемонстрував квантові алгоритми, здатні розв'язувати ці задачі за 
поліноміальний час на достатньо потужному квантовому комп'ютері. Це створює критичну загрозу для існуючої криптографічної 
інфраструктури.

Постквантова криптографія (Post-Quantum Cryptography, PQC) -- це галузь криптографії, що розробляє алгоритми, стійкі 
як до класичних, так і до квантових атак. Серед основних напрямків PQC виділяють криптографію на решітках, криптографію 
на кодах виправлення помилок, багатовимірну поліноміальну(квадратичну) криптографію та криптографію на основі 
геш-функцій~\cite{WikiPQC}.

\section{Передумови створення TiGER}
\label{sec:2-2}

Механізм інкапсуляції ключа (Key Encapsulation Mechanism, KEM) є одним з найважливіших криптографічних примітивів для 
захищеного обміну ключами. У контексті заміни класичних протоколів, таких як Diffie-Hellman (DH) або Elliptic Curve 
Diffie-Hellman ECDH, постквантові KEM повинні забезпечувати не лише високий рівень безпеки, але й бути ефективними за 
розміром даних та залишатися обчислювано складними для зламу зловмисником.

Криптографія на решітках, зокрема алгоритми на основі задач Learning With Errors (LWE)~\cite{regev2005} та Ring Learning 
With Errors (RLWE)~\cite{lyubashevsky2010}, продемонструвала перспективність у створенні ефективних постквантових 
схем. Розвиток цього напрямку призвів до появи сімейства алгоритмів, що використовують детерміністичний варіант -- 
Learning With Rounding (LWR)~\cite{banerjee2012}, який замінює випадкову помилку округленням, що покращує як 
продуктивність, так і довжину шифротексту.

Серед попередніх розробок слід відзначити алгоритми Lizard~\cite{lizard2018} та RLizard~\cite{rlizard2018}, які 
комбінували RLWE для генерації ключів з RLWR для шифрування, досягаючи балансу між безпекою та ефективністю. Однак 
ці схеми мали певні обмеження щодо розміру відкритого ключа та шифротексту, що ускладнювало їх інтеграцію в існуючі протоколи.

TiGER (Tiny bandwidth key encapsulation mechanism for easy miGration based on RLWE(R))~\cite{tiger2022} був розроблений 
командою дослідників з метою створення компактного та ефективного KEM, придатного для легкої інтеграції в існуючі системи 
безпеки. Основні задачі, які ставили перед собою науковці це:

\begin{itemize}
    \item \textbf{Мінімізація розміру шифротексту та відкритого ключа}
    \item \textbf{Висока обчислювальна ефективність} --- використання в якості модуля число, яке є степенем двійки ($q = 2^{k}$) 
        (для оптимізації операцій округлення через побітові зсуви);
    \item \textbf{Відмова від NTT} --- алгоритм не використовує Number Theoretic Transform, що спрощує реалізацію;
    \item \textbf{Використання розріджених секретів} --- зменшення розміру секретного ключа та прискорення 
        множення многочленів;
    \item \textbf{Корекція помилок} --- застосування кодів XEf та D2 для зниження ймовірності помилки дешифрування.
\end{itemize}

Конструкція TiGER базується на комбінації RLWR для генерації відкритого ключа та RLWE для шифрування, з подальшим застосуванням перетворення Fujisaki-Okamoto \cite{fo1999, fo2013} для досягнення IND-CCA безпеки.

\section{Участь у KpqC та злиття з SMAUG}
\label{sec:2-3}

У 2022 році Національна служба розвідки Республіки Корея ініціювала Korean Post-Quantum Cryptography Competition 
скорочено -- KpqC~\cite{kpqc2023}. Це національний конкурс для стандартизації постквантових криптографічних алгоритмів.

Обраний нами для аналізу алгоритм TiGER був поданий на перший раунд конкурсу KpqC у категорії механізмів інкапсуляції 
ключа (KEM) і був одним з чотирьох алгоритмів, які пройшли до другого раунду.

\subsection*{Злиття TiGER та SMAUG}
Команди TiGER та SMAUG об'єдналися для створення спільного алгоритму SMAUG-T~\cite{smaugt2024}. Метою злиття було 
поєднання переваг обох підходів:

\begin{itemize}
    \item Від \textbf{TiGER}: Компактність шифротексту, використання RLWE/RLWR на кільцевому рівні, корекція помилок 
        через D2 кодування (для параметра TiMER);
    \item Від \textbf{SMAUG}: Модульна структура (MLWE/MLWR), розріджені секрети через використання гаусівського 
        шуму, покращена безпека за рахунок збільшення розмірності.
\end{itemize}

Результатом злиття став алгоритм SMAUG-T версії 3.0 (лютий 2024), який включає в себе:
\begin{itemize}
    \item Три основні набори параметрів: \textbf{SMAUG-T128}, \textbf{SMAUG-T192}, \textbf{SMAUG-T256} 
        (відповідають рівням безпеки NIST 1, 3, 5);
    \item Додатковий набір параметрів \textbf{TiMER} (Tiny SMAUG using Error Reconciliation) -- оптимізований для 
        IoT-пристроїв з мінімальним шифротекстом завдяки використанню D2 кодування з TiGER.
\end{itemize}

\subsection*{Результати KpqC 2023}

У січні 2025 року було оголошено фінальні результаті конкурсу KpqC. Переможцями стали:
\begin{itemize}
    \item У категорії KEM: \textbf{SMAUG-T} та \textbf{NTRU+};
    \item У категорії цифрового підпису: \textbf{HAETAE} (до речі, також від команди SMAUG).
\end{itemize}

Таким чином, ідеї та технології TiGER увійшли до складу національного стандарту постквантової криптографії Кореї 
через алгоритм SMAUG-T.

\chapter{\large{Теоретична база алгоритму TiGER}}
\label{chapter-2}

% TODO: Заповнити розділ

\chapter{\large{Повний опис алгоритму TiGER}}
\label{chapter-3}

% TODO: Заповнити розділ

\chapter{\large{Результати досліджень}}
\label{chapter-4}

% TODO: Заповнити розділ

\chapter{\large{Аналіз атак на TiGER}}
\label{chapter-5}

% TODO: Заповнити розділ

\chapter{\large{Порівняльний аналіз}}
\label{chapter-6}

% TODO: Заповнити розділ

\chapter{\large{Перенесення атак та можливі покращення}}
\label{chapter-7}

% TODO: Заповнити розділ
